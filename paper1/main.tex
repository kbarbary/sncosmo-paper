%% This is file `elsarticle-template-2-harv.tex',
%%
%% Copyright 2009 Elsevier Ltd
%%
%% This file is part of the 'Elsarticle Bundle'.
%% ---------------------------------------------
%%
%% It may be distributed under the conditions of the LaTeX Project Public
%% License, either version 1.2 of this license or (at your option) any
%% later version.  The latest version of this license is in
%%    http://www.latex-project.org/lppl.txt
%% and version 1.2 or later is part of all distributions of LaTeX
%% version 1999/12/01 or later.
%%
%% The list of all files belonging to the 'Elsarticle Bundle' is
%% given in the file `manifest.txt'.
%%
%% Template article for Elsevier's document class `elsarticle'
%% with harvard style bibliographic references
%%
%% $Id: elsarticle-template-2-harv.tex 155 2009-10-08 05:35:05Z rishi $
%% $URL: http://lenova.river-valley.com/svn/elsbst/trunk/elsarticle-template-2-harv.tex $
%%
\documentclass[5p,authoryear]{elsarticle}

\usepackage{color}
\usepackage{aas_macros}

%% Use the option review to obtain double line spacing
%% \documentclass[authoryear,preprint,review,12pt]{elsarticle}

%% Use the options 1p,twocolumn; 3p; 3p,twocolumn; 5p; or 5p,twocolumn
%% for a journal layout:
%% \documentclass[final,authoryear,1p,times]{elsarticle}
%% \documentclass[final,authoryear,1p,times,twocolumn]{elsarticle}
%% \documentclass[final,authoryear,3p,times]{elsarticle}
%% \documentclass[final,authoryear,3p,times,twocolumn]{elsarticle}
%% \documentclass[final,authoryear,5p,times]{elsarticle}
%% \documentclass[final,authoryear,5p,times,twocolumn]{elsarticle}

%% if you use PostScript figures in your article
%% use the graphics package for simple commands
%% \usepackage{graphics}
%% or use the graphicx package for more complicated commands
%% \usepackage{graphicx}
%% or use the epsfig package if you prefer to use the old commands
%% \usepackage{epsfig}

%% The amssymb package provides various useful mathematical symbols
\usepackage{amssymb}
%% The amsthm package provides extended theorem environments
%% \usepackage{amsthm}

%% The lineno packages adds line numbers. Start line numbering with
%% \begin{linenumbers}, end it with \end{linenumbers}. Or switch it on
%% for the whole article with \linenumbers after \end{frontmatter}.
%% \usepackage{lineno}

%% natbib.sty is loaded by default. However, natbib options can be
%% provided with \biboptions{...} command. Following options are
%% valid:

%%   round  -  round parentheses are used (default)
%%   square -  square brackets are used   [option]
%%   curly  -  curly braces are used      {option}
%%   angle  -  angle brackets are used    <option>
%%   semicolon  -  multiple citations separated by semi-colon (default)
%%   colon  - same as semicolon, an earlier confusion
%%   comma  -  separated by comma
%%   authoryear - selects author-year citations (default)
%%   numbers-  selects numerical citations
%%   super  -  numerical citations as superscripts
%%   sort   -  sorts multiple citations according to order in ref. list
%%   sort&compress   -  like sort, but also compresses numerical citations
%%   compress - compresses without sorting
%%   longnamesfirst  -  makes first citation full author list
%%
%% \biboptions{longnamesfirst,comma}

% \biboptions{}

\journal{Astronomy \& Computing}

\begin{document}

\begin{frontmatter}

%% Title, authors and addresses

%% use the tnoteref command within \title for footnotes;
%% use the tnotetext command for the associated footnote;
%% use the fnref command within \author or \address for footnotes;
%% use the fntext command for the associated footnote;
%% use the corref command within \author for corresponding author footnotes;
%% use the cortext command for the associated footnote;
%% use the ead command for the email address,
%% and the form \ead[url] for the home page:
%%
%% \title{Title\tnoteref{label1}}
%% \tnotetext[label1]{}
%% \author{Name\corref{cor1}\fnref{label2}}
%% \ead{email address}
%% \ead[url]{home page}
%% \fntext[label2]{}
%% \cortext[cor1]{}
%% \address{Address\fnref{label3}}
%% \fntext[label3]{}

\title{SNCosmo}

%% use optional labels to link authors explicitly to addresses:
%% \author[label1,label2]{<author name>}
%% \address[label1]{<address>}
%% \address[label2]{<address>}

\author[ucb]{K.~Barbary}
\ead{kbarbary@berkeley.edu}

\address[ucb]{Berkeley, CA}
\begin{abstract}
%- What is SNCosmo 
We present SNCosmo, an open source, well-documented Python library under 
continuous development by users, providing functions for the basic tasks in 
supernova cosmology ({\color{red}{Should there be any clear definition of the scope/vision 
for SNCosmo  eg. will not do image reduction?})} 
%- Who should use it and what can it be used for
The library is intended to be used by the researcher performing a sequence 
of tasks in an analysis pipeline combining user-defined-functions with calls to 
the SNCosmo library functions, as well as by the user who wishes to perform 
individual tasks. 
%- Current Features: 
%* Versatile API for new supernova light curve models, with popular models like 
% SALT2 (and HSIAO model) implemented as examples 
%* Implemented models: SALT2, HSIAO
%* Simulation of data from light curve models
%* Faciliate exploration of supernova data
%* Multiple methods of btaining SNIa model parameters from SNIa light curves
%* Interaction with outputs of other software
Currently, the library features a versatile API for light curve model with 
example implementations of the popular SALT2 model, and methods for obtaining 
parameters of such models from supernova light curves. This allows one to 
simulate supernova data from survey characteristics, fit simulated or real 
data to light curve models, and provides an output format and plotting functionality for easy exploration and analysis of data. ({\color{red}{Do we think that it would 
attract more users if the functionality could be expanded to obtaining cosmology?)}} The library is easy to install, available publicly at github.com/sncosmo/sncosmo and the documentation provides  example How-Tos to achieve a number of tasks. 

In this document, we provide a brief motivation, comparison with existing 
software. Contributions and suggestions from the user community are welcome ...
\end{abstract}


\begin{keyword}
%% keywords here, in the form: keyword \sep keyword

%% MSC codes here, in the form: \MSC code \sep code
%% or \MSC[2008] code \sep code (2000 is the default)

\end{keyword}

\end{frontmatter}

% \linenumbers

%% main text
\section{Introduction}
\label{intro}

This is the introduction. We haven't cited anything yet, but if we did, we might cite \citet{guy07a} or not \citep{guy05a}.

\section{Light Curve Parameter Posterior Estimation}

%\documentclass{article}[12pt]
%\input{preamble}
\begin{document}
\section{Introduction}
\begin{itemize}
\item SN Cosmology based on maps of Light curve properties to intrinsic luminosities of SN
\item Usually done by a light curve model with free parameters, determined for SN in question, and a map from these free parameters to brightness  
\item How to obtain such free parameters of the model given photometric data, and account for uncertainties in these model parameters.
\end{itemize}
\section{Question: Does sampling of LC model posteriors improve SNIa cosmology?}
\begin{itemize}
\item  This would be obvious if we were using samples to do cosmology. Right
now we do not do this. Can we say that sampling is a better idea even in the 
absence of a method using full posteriors for cosmology?
\item We could hope for two things: First, better estimates of errors/covariances. Second, better estimates of reported values.
\item The first would happen, if, for example, the likelihoods differ from 
Gaussian likelihoods as we move away from the ML point. The usual ML procedure
would mis-estimate the errors involved on this. (MINOS probably would not, but
that is rarely done). So, one could ask how frequently this happens for
different kinds of data. 
\item The second would happen if the ML procedure gets stuck in local 
likelihood peaks. 
\end{itemize}
\subsection{How do we do try to answer these questions}
\begin{itemize}
\item Use a simulation, where the input is well described by the model we are 
using. This means ``no zero point smearing``, no ``intrinsic scatter model`` 
from SNANA, for a survey. Perhaps allow for floating $z$? Validation of simulations by showing relevant quantities like pulls 
are good for simulation. Number of SN simulated must be large, because the cases of importance might be rare.
\item Repeat with ML estimator.  Find `outliers`. 
\item Repeat with sampler. Find `outliers`. 
\end{itemize}
\section{How good are the samplers we are using?}
\end{document}


%% The Appendices part is started with the command \appendix;
%% appendix sections are then done as normal sections
%% \appendix

%% \section{}
%% \label{}

%% References
%%
%% Following citation commands can be used in the body text:
%%
%%  \citet{key}  ==>>  Jones et al. (1990)
%%  \citep{key}  ==>>  (Jones et al., 1990)
%%
%% Multiple citations as normal:
%% \citep{key1,key2}         ==>> (Jones et al., 1990; Smith, 1989)
%%                            or  (Jones et al., 1990, 1991)
%%                            or  (Jones et al., 1990a,b)
%% \cite{key} is the equivalent of \citet{key} in author-year mode
%%
%% Full author lists may be forced with \citet* or \citep*, e.g.
%%   \citep*{key}            ==>> (Jones, Baker, and Williams, 1990)
%%
%% Optional notes as:
%%   \citep[chap. 2]{key}    ==>> (Jones et al., 1990, chap. 2)
%%   \citep[e.g.,][]{key}    ==>> (e.g., Jones et al., 1990)
%%   \citep[see][pg. 34]{key}==>> (see Jones et al., 1990, pg. 34)
%%  (Note: in standard LaTeX, only one note is allowed, after the ref.
%%   Here, one note is like the standard, two make pre- and post-notes.)
%%
%%   \citealt{key}          ==>> Jones et al. 1990
%%   \citealt*{key}         ==>> Jones, Baker, and Williams 1990
%%   \citealp{key}          ==>> Jones et al., 1990
%%   \citealp*{key}         ==>> Jones, Baker, and Williams, 1990
%%
%% Additional citation possibilities
%%   \citeauthor{key}       ==>> Jones et al.
%%   \citeauthor*{key}      ==>> Jones, Baker, and Williams
%%   \citeyear{key}         ==>> 1990
%%   \citeyearpar{key}      ==>> (1990)
%%   \citetext{priv. comm.} ==>> (priv. comm.)
%%   \citenum{key}          ==>> 11 [non-superscripted]
%% Note: full author lists depends on whether the bib style supports them;
%%       if not, the abbreviated list is printed even when full requested.
%%
%% For names like della Robbia at the start of a sentence, use
%%   \Citet{dRob98}         ==>> Della Robbia (1998)
%%   \Citep{dRob98}         ==>> (Della Robbia, 1998)
%%   \Citeauthor{dRob98}    ==>> Della Robbia


%% References with bibTeX database:

\bibliographystyle{elsarticle-harv}
\bibliography{main.bib}

%% Authors are advised to submit their bibtex database files. They are
%% requested to list a bibtex style file in the manuscript if they do
%% not want to use model2-names.bst.

%% References without bibTeX database:

% \begin{thebibliography}{00}

%% \bibitem must have one of the following forms:
%%   \bibitem[Jones et al.(1990)]{key}...
%%   \bibitem[Jones et al.(1990)Jones, Baker, and Williams]{key}...
%%   \bibitem[Jones et al., 1990]{key}...
%%   \bibitem[\protect\citeauthoryear{Jones, Baker, and Williams}{Jones
%%       et al.}{1990}]{key}...
%%   \bibitem[\protect\citeauthoryear{Jones et al.}{1990}]{key}...
%%   \bibitem[\protect\astroncite{Jones et al.}{1990}]{key}...
%%   \bibitem[\protect\citename{Jones et al., }1990]{key}...
%%   \harvarditem[Jones et al.]{Jones, Baker, and Williams}{1990}{key}...
%%

% \bibitem[ ()]{}

% \end{thebibliography}

\end{document}

%%
%% End of file `elsarticle-template-2-harv.tex'.
