
\begin{itemize}
\item SN Cosmology based on maps of Light curve properties to intrinsic luminosities of SN
\item Usually done by a light curve model with free parameters, determined for SN in question, and a map from these free parameters to brightness  
\item How to obtain such free parameters of the model given photometric data, and account for uncertainties in these model parameters.
\end{itemize}

\subsection{Question: Does sampling of LC model posteriors improve SNIa cosmology?}

\begin{itemize}
\item  This would be obvious if we were using samples to do cosmology. Right
now we do not do this. Can we say that sampling is a better idea even in the 
absence of a method using full posteriors for cosmology?
\item We could hope for two things: First, better estimates of errors/covariances. Second, better estimates of reported values.
\item The first would happen, if, for example, the likelihoods differ from 
Gaussian likelihoods as we move away from the ML point. The usual ML procedure
would mis-estimate the errors involved on this. (MINOS probably would not, but
that is rarely done). So, one could ask how frequently this happens for
different kinds of data. 
\item The second would happen if the ML procedure gets stuck in local 
likelihood peaks. 
\end{itemize}

\subsubsection{How do we do try to answer these questions}
\begin{itemize}
\item Use a simulation, where the input is well described by the model we are 
using. This means ``no zero point smearing``, no ``intrinsic scatter model`` 
from SNANA, for a survey. Perhaps allow for floating $z$? Validation of simulations by showing relevant quantities like pulls 
are good for simulation. Number of SN simulated must be large, because the cases of importance might be rare.
\item Repeat with ML estimator.  Find `outliers`. 
\item Repeat with sampler. Find `outliers`. 
\end{itemize}

\subsection{How good are the samplers we are using?}

