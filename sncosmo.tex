\documentclass{emulateapj}


% has to be before amssymb it seems
%\usepackage{color,hyperref}
%\definecolor{linkcolor}{rgb}{0,0,0.5}
%\hypersetup{colorlinks=true,linkcolor=linkcolor,citecolor=linkcolor,
%            filecolor=linkcolor,urlcolor=linkcolor}
%\usepackage{amssymb,amsmath}

\usepackage{color}
\usepackage{url}
\usepackage{graphicx}
\graphicspath{{figures/}}

% For Python code
\usepackage{listings}
\definecolor{lbcolor}{rgb}{0.9,0.9,0.9}
\lstset{language=Python,
        basicstyle=\footnotesize\ttfamily,
        showspaces=false,
        showstringspaces=false,
        tabsize=2,
        breaklines=false,
        breakatwhitespace=true,
        identifierstyle=\ttfamily,
        keywordstyle=\bfseries\color[rgb]{0.133,0.545,0.133},
        commentstyle=\color[rgb]{0.133,0.545,0.133},
        stringstyle=\color[rgb]{0.627,0.126,0.941},
    }

% Draft watermark:
\usepackage{draftwatermark}
\SetWatermarkLightness{0.9}
\SetWatermarkScale{4}

\usepackage[normalem]{ulem}
\newcommand{\new}[1]{{\color{red} #1}}
\newcommand{\old}[1]{{\sout{#1}}}


\begin{document}

\title{SNCosmo}

\newcommand{\berkeley}{1}
\newcommand{\uwastro}{2}
\newcommand{\argonne}{3}
\author{Kyle Barbary\altaffilmark{\berkeley}}
\author{Rahul Biswas\altaffilmark{\uwastro}}
\author{Argonneians\altaffilmark{\argonne}}
\altaffiltext{\berkeley}{Department of Physics, University of California, Berkeley}
\altaffiltext{\uwastro}{Department of Astronomy, University of Washington}
\altaffiltext{\argonne}{Argonne National Laboratory}


\begin{abstract}
We present SNCosmo, an open-source Python library for common analysis
tasks in supernova cosmology.  The library is intended to be used by
the researcher performing a sequence of tasks in an analysis pipeline
combining user-defined-functions with calls to the SNCosmo library, as
well as by the user who wishes to perform quick
calculations. Currently, the library features a flexible API for
supernova light curve models that provides a common interface for
multiple types of models. Popular Type Ia and core-collapse models
such as SALT2, Hsiao, and Nugent templates are built in and the system
is designed to be extended to more complex models with varying numbers
of parameters.  Functionality in the library allows one to use any
model to simulate supernova data from survey characteristics, or fit a
model to simulated or real photometric data using a range of fitting
functionality.  Also included are convenience functions for input and
output data formatting and plotting for easy exploration and analysis
of data. The library is easy to install, publicly available and
licensed under a liberal BSD-style license. The API is well-documented
and the online documentation provides examples to achieve a number of
tasks. SNCosmo is under continuous development by users and
contributions from the community are welcome.
\end{abstract}

\keywords{
    methods: data analysis ---
    methods: statistical
}

\section{Introduction}
\label{introduction}

We haven't cited anything yet, but if we did, we might cite
\citet{guy07a} or not \citep{guy05a}.

In this document, we provide a brief motivation, comparison with
existing software. Contributions and suggestions from the user
community are welcome.


\section{Current Features}
\label{features}


\begin{itemize}
\item Versatile API for new supernova light curve models, with popular models like SALT2 (and HSIAO model) implemented as examples 
\item Implemented models: SALT2, HSIAO
\item Simulation of data from light curve models
\item Faciliate exploration of supernova data
\item Multiple methods of btaining SNIa model parameters from SNIa light curves
\item Interaction with outputs of other software
\end{itemize}


\bibliographystyle{apj}
\bibliography{sncosmo}

\end{document}
